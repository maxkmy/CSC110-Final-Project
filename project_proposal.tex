\documentclass[fontsize=11pt]{article}
\usepackage{amsmath}
\usepackage[utf8]{inputenc}
\usepackage[margin=0.75in]{geometry}
\usepackage[shortlabels]{enumitem}

\title{CSC110 Project Proposal: Global Wealth Gap}
\author{Max Ming Yi Koh, Kevin Trung Le, Kai Jun Zhuang, Zi Kai Xu}
\date{Friday, November 5, 2021}

\begin{document}
\maketitle

\section*{Problem Description and Research Question}
Research conducted by economists shows the wealth gap widened during the industrial revolution (Rychbosh, 2015). Before the industrial revolution, the richest countries were 3 times as rich as the poorest countries. Today, this difference widens to more than 100-fold. The wealth gap deserves attention as poorer countries are more susceptible to social problems like poverty, terrorism and subpar living condition (Krieger, 2016; Cookson, 2020).
\\

\noindent Two ideas come to mind when considering how COVID-19 and the global wealth gap interact:

\begin{enumerate}
    \item COVID-19 has varying degrees of impact on different industries. For instance, in 2020, global tourism lost \$935 billion, a 10.5\% reduction compared to 2019’s revenue (Madden, 2021). Relatively, the impact on manufacturing sector is lesser where production declined 7\% globally in 2020 (Todorov, 2020).

    Moreover, countries differ in economic structures. Using the three-sector model, the economies of developed nations are mostly made up of tertiary sector (services) and secondary sector (manufacturing), while undeveloped nations depend on secondary sector and primary sector (extraction and agriculture).

    Thus, it is interesting to understand the magnitude of impact COVID-19 exerts on different nations of different economic structures.

    \item Research shows 60\% of people in high-income countries receive a first dose vaccine while this number is only 1\% in low-income countries (N/A, 2021). This disparity is attributable to different levels of wealth and healthcare networks which affects countries’ ability to respond to COVID-19.

    We predict that quicker response of wealthier nations resulted in less impact on their workforce and economy, thus widening the wealth gap. We hope to examine this prediction using datasets we find.
\end{enumerate}

\\
\noindent The 2 ideas lead us to question  \textbf{how COVID-19 impacts the global wealth gap}. Since gross domestic product (GDP) is a widely accepted economic metric to measure a nation’s output and income, we will analyze GDP by sector, GDP and unemployment rate to assess the magnitude of impact COVID-19 exerts on countries of different stages of development (idea 1). We also consider non-economic metrics like death rate and vaccination rate to understand how nations’ varying capabilities to respond to COVID-19 influence their economic performance (idea 2).  \\

\noindent Overall, these findings will help understand the impact COVID-19 exerts on countries from economic and social perspectives. The results may help policy makers at national and international levels (e.g. World Bank) better understand the vulnerability of economies of different nations. Appropriate recommendations can be made to reduce the wealth gap, thus better preparing for future pandemics.
\newpage

\section*{Dataset Description}
\begin{enumerate}
    \item \textbf{Mortality}
        \begin{enumerate}[(a)]
            \item \textbf{Source}: World Happiness Report
            \item \textbf{Format}: CSV
            \item \textbf{Relevant Column Headers}: Country name, COVID-19 deaths per 100,000 population in 2020
            \item \textbf{Dimensions}: 167 rows and 4 columns
            \item \textbf{Additional Notes}: None
        \end{enumerate}
    \item \textbf{Annual GDP}
        \begin{enumerate}[(a)]
            \item \textbf{Source}: World Bank
            \item \textbf{Format}: CSV
            \item \textbf{Relevant Column Headers}: Country Name, 2016 GDP, 2017 GDP, 2018 GDP, 2019 GDP, 2020 GDP
            \item \textbf{Dimensions}: 271 rows and 65 columns
            \item \textbf{Additional Notes}: Monetary values are recorded in USD.
        \end{enumerate}
    \item \textbf{Country Income Quartile}
        \begin{enumerate}[(a)]
            \item \textbf{Source}: World Bank
            \item \textbf{Format}: CSV
            \item \textbf{Relevant Column Headers}: Country Code, Income Group
            \item \textbf{Dimensions}: 273 rows and 4 columns
            \item \textbf{Additional Notes}: Income group is separated into 4 quartiles (high, upper middle, lower middle, low).
        \end{enumerate}
    \item \textbf{GDP By Sector}
        \begin{enumerate}[(a)]
            \item \textbf{Source}: World Bank
            \item \textbf{Format}: CSV
            \item \textbf{Relevant Column Headers}: Country Name, Manufacturing 2016, Service 2016, Industry 2016, Agriculture, forestry and fishing 2016 (same format for 2017, 2018, 2019, 2020)
            \item \textbf{Dimensions}: 272 rows and 22 columns
            \item \textbf{Additional Notes}: Monetary values are recorded in USD.
        \end{enumerate}
    \item \textbf{Unemployment Rate}
        \begin{enumerate}[(a)]
            \item \textbf{Source}: World Bank
            \item \textbf{Format}: CSV
            \item \textbf{Relevant Column Headers}: Country Name, 2016 Unemployment as \% of Workforce, 2017 Unemployment as \% of Workforce, 2018 Unemployment as \% of Workforce, 2019 Unemployment as \% of Workforce, 2020 Unemployment as \% of Workforce
            \item \textbf{Dimensions}: 271 rows and 65 columns
            \item \textbf{Additional Notes}: None
        \end{enumerate}
    \item \textbf{Vaccination}
        \begin{enumerate}[(a)]
            \item \textbf{Source}: Our World In Data
            \item \textbf{Format}: CSV
            \item \textbf{Relevant Column Headers}: Total Vaccinations as a Percentage of the Population
            \item \textbf{Dimensions}: 226 rows and 63 columns
            \item \textbf{Additional Notes}: None
        \end{enumerate}
\end{enumerate}

\section*{Computational Plan}
    \subsection*{Data Processing}
    \begin{enumerate}
        \item Create a \texttt{Country} class that bundles data in different CSV files together. Instance attributes include country, 2016 - 2020 GDP, 2016 - 2020 GDP by sector, income quartile, unemployment rate, COVID-19 death rate and vaccination rate.
        \item Create a dictionary that maps a country name to a \texttt{Country} instance with attributes initialized as \texttt{None}.
        \item Extract necessary fields from each CSV file and assign them to appropriate instance attributes.
        \item The dictionary now maps a country's name to its \texttt{Country} instance containing the attributes required for computation. We refer to this dictionary as \texttt{country\_dict}.
    \end{enumerate}

    \subsection*{Required Libraries}
    \begin{enumerate}
        \item \textbf{Plotly}: Plotly plots data into interactive graphs. Plotly will be used to plot line graphs for chronological data, bar charts for qualitative data and scatter plots and trendlines to verify metrics' correlation.
        \item \textbf{Geopandas}: Geopandas helps work with geospatial data. Geopandas will be used to plot choropleth maps which are coloured maps where the colour of a country is determined by a numeric value.
    \end{enumerate}

    \subsection*{Computation By Metrics}
    \begin{enumerate}
        \item \textbf{GDP \% Change}
            \begin{enumerate}[(a)]
                \item \textbf{Formula}: Year $n$ GDP \% Change $= \frac{\text{GDP}_{n} \ - \ \text{GDP}_{n-1}}{\text{GDP}_{n-1}} \times 100$
                \item \textbf{Computation}: For years 2016 to 2020, iterate over all countries in \texttt{country\_dict} and apply the formula for pairs of consecutive years.
                \item \textbf{Visualization}: Generate a chloropleth map from \% change in GDP by year (e.g. green represents high GDP growth) using \texttt{Geopandas}. Plot GDP \% change over time as line graphs using \texttt{Plotly}.
            \end{enumerate}

        \item \textbf{GDP as a \% of Global GDP}
            \begin{enumerate}[(a)]
                \item \textbf{Formula}: GDP of Country as \% of Global GDP $= \frac{\text{National GDP}}{\text{Global GDP}} \times 100$
                \item \textbf{Computation}: For years 2016 to 2020, iterate over countries in \texttt{country\_dict} to aggregate GDP to find global GDP. Iterate through all the countries in \texttt{country\_dict} again to use the country's GDP and the global GDP and apply the formula.
                \item \textbf{Visualization}: Similar to \texttt{GDP \% Change}
            \end{enumerate}
        \item \textbf{Unemployment Rate \% Change}
            \begin{enumerate}[(a)]
                \item \textbf{Formula}: Year $n$ Unemployment Rate \% Change $= \frac{\text{Unemployment \ Rate}_{n}-\  \text{Unemployment \ Rate}_{n-1}}{\text{Unemployment \ Rate}_{n-1}} \times 100$
                \item \textbf{Computation}: Iterate over all countries in \texttt{country\_dict} and apply the formula for pairs of consecutive years from 2016 to 2020.
                \item \textbf{Visualization}: Similar to \texttt{GDP \% Change}.
            \end{enumerate}
        \item \textbf{Aggregate Sector GDP as a \% of Aggregate GDP Grouped by National Income Quartile}
            \begin{enumerate}[(a)]
                \item \textbf{Formula}: Aggregate Sector GDP as \% of Aggregate GDP $= \frac{\text{Aggregate \ Sector \ GDP}}{\text{Aggregate \ GDP}} \times 100$
                \item \textbf{Computation}: Iterate over \texttt{country\_dict} and aggregate national GDP of all countries in the desired quartile to calculate aggregate GDP. Aggregate sector GDP for nations in the desired quartile to calculate aggregate sector GDP. Apply the formula with the values calculated.
                \item \textbf{Visualization}: Generate a bar chart using \texttt{Plotly} for the four sectors (from dataset). For each sector, plot \% of aggregate GDP by income quartile.
            \end{enumerate}

        \newpage

        \item \textbf{COVID-19 related deaths as a predictor of National GDP}
            \begin{enumerate}[(a)]
                \item \textbf{Formula}: GDP = $\beta_{0} + \beta_{1} \times$ COVID-19 deaths per 100,000 population in 2020 $+ \  \epsilon_{i}$
                \item \textbf{Computation}: Using linear regression, compute the fitted linear regression line and regression coefficients to determine an equation that predicts national GDP given the COVID-19 deaths per 100,000 population in 2020 of a country.
                \item \textbf{Visualization}: Create a scatter plot with each point representing a country on an x-axis of COVID-19 deaths per 100,000 population in 2020 and a y-axis of National GDP and plot the fitted linear regression line using \texttt{Plotly}.
            \end{enumerate}
        \item \textbf{Total Vaccinations as a Percentage of the Population as a predictor of National GDP}
            \begin{enumerate}[(a)]
                \item \textbf{Formula}: GDP = $\beta_{0} + \beta_{1} \times$ Total Vaccinations as a Percentage of the Population $+ \ \epsilon_{i}$
                \item \textbf{Computation}: Similar to \texttt{COVID-19 related deaths as a predictor of National GDP}
                \item \textbf{Visualization}: Similar to \texttt{COVID-19 related deaths as a predictor of National GDP}
            \end{enumerate}
    \end{enumerate}

\newpage
\begin{center}
\section*{References}
\end{center}

    \item All Countries and Economies GDP (current US\$). World Bank Open Data . (2020). Retrieved November 4, 2021, \indent from https://data.worldbank.org/indicator/NY.GDP.MKTP.CD?view=chart.  \\

    \item Development Indicators. World Bank DataBank. (2021, October 28). Retrieved November 4, 2021, from \\ \indent https://databank.worldbank.org/source/world-development-indicators?l=en. \\

    \item Helliwell, J., Layard, R., Sachs, J. D., Neve, J.-E. D., Aknin, L., Wang, S., &amp; Paculor, S. (2021). World happiness \indent report 2021. World Happiness Report 2021. Retrieved November 4, 2021, from https://worldhappiness.report/ed/\\ \indent 2021/.\\

    \item Krieger, T.,  Meierrieks, D. (2016, March).  Does Income Inequality Lead to Terrorism?  (Rep.  No.  5821). Retrieved \indent from https://www.ifo.de/DocDL/cesifo1wp5821.pdf \\

    \item Madden, D. (2021, January 14). The COVID-19 pandemic has cost the global tourism industry \$935 billion. \indent Forbes. Retrieved November 4, 2021, from https://www.forbes.com/sites/duncanmadden/2021/01/14/the- covid-\indent 19-pandemic-has-cost-the-global-tourism-industry-935-billion/?sh=1de6c2077d40. \\

    \item Ritchie, H., Mathieu, E., Rodés-Guirao, L., Appel, C., Giattino, C., Ortiz-Ospina, E., Hasell, J., Macdonald, B., \indent Beltekian, D., &amp; Roser, M. (2020, March 5). Coronavirus (COVID-19) vaccinations - statistics and research. \indent Our World in Data. Retrieved November 4, 2021, from https://ourworldindata.org/covid-vaccinations. \\

    \item Ryckbosch, W. (2015, September 25). Economic inequality and growth before the industrial revolution: the case \indent of the Low Countries (fourteenth to nineteenth centuries). Oxford Academic. Retrieved November 4, 2021, from \indent https://academic.oup.com/ereh/article/20/1/1/2465267. \\

    \item The Economist Newspaper. (2021, August 30). Vaccine inequality will cost money as well as lives. The Economist. \indent Retrieved November 4, 2021, from https://www.economist.com/graphic-detail/2021/08/30/vaccine-inequality-\indent will-cost-money-as-well-as-lives. \\

    \item Todorov, V. (2020, June 3). Global manufacturing production drops sharply due to economic disruptions caused \indent by covid-19 – UNIDO report. UNIDO. Retrieved November 4, 2021, from https://www.unido.org/news/global-\indent manufacturing-production-drops-sharply-due-economic-disruptions-caused-covid-19-unido-report. \\

    \item Understanding Poverty. World Bank. (2021, October 14). Retrieved November 4, 2021, from \\ \indent https://www.worldbank.org/en/topic/poverty/overview#1. \\

    \item Unemployment, total (\% of total labor force) (modeled ILO estimate). World Bank Open Data. (2020). Retrieved \indent November 4, 2021, from https://data.worldbank.org/indicator/SL.UEM.TOTL.ZS. \\

\end{document}
