\documentclass[fontsize=11pt]{article}
\usepackage{amsmath}
\usepackage[utf8]{inputenc}
\usepackage[margin=0.75in]{geometry}
\usepackage[shortlabels]{enumitem}

\title{CSC110 Project Proposal: Global Wealth Gap}
\author{Max Ming Yi Koh, Kevin Trung Le, Kai Jun Zhuang, Zi Kai Xu}
\date{Friday, November 5, 2021}

\begin{document}
\maketitle

\section*{Problem Description and Research Question}
Research conducted by economists shows the wealth gap widened during the industrial revolution (Rychbosh, 2015). Before the industrial revolution, the richest countries were 3 times as rich as the poorest countries. Today, this difference widens to more than 100-fold. The wealth gap deserves attention as poorer countries are more susceptible to social problems like poverty, terrorism and subpar living condition (Krieger, 2016; Cookson, 2020).
% TODO: in-text citation for last line might change
% due to formatting or Kevin potentially not finding the original link
\\

\noindent Two ideas come to mind when considering how COVID-19 and the global wealth gap interact:

\begin{enumerate}
    \item COVID-19 has varying degrees of impact on different industries. For instance, in 2020, global tourism lost \$935 billion, a 10.5\% reduction compared to 2019’s revenue. Relatively, the impact on manufacturing sector is lesser where production declined 7\% globally in 2020 (Todorov, N/A).

    Moreover, countries differ in economic structures. Using the three-sector model, the economies of developed nations are mostly made up of tertiary sector (services) and secondary sector (manufacturing), while undeveloped nations depend on secondary sector and primary sector (extraction and agriculture).

    Thus, it is interesting to understand the magnitude of impact COVID-19 exerts on different nations of different economic structures.

    \item Research shows 60\% of people in high-income countries receive a first dose vaccine while this number is only 1\% in low-income countries (N/A, 2021). This disparity is attributable to different levels of wealth and healthcare networks which affects countries’ ability to respond to COVID-19.

    We predict that quicker response of wealthier nations resulted in less impact on their workforce and economy, thus widening the wealth gap. We hope to examine this prediction using datasets we find.
\end{enumerate}

\\
\noindent The 2 ideas lead us to question  \textbf{how COVID-19 impacts the global wealth gap}. Since gross domestic product (GDP) is a widely accepted economic metric to measure a nation’s output and income, we will analyze GDP by sector, GDP and unemployment rate to assess the magnitude of impact COVID-19 exerts on countries of different stages of development (idea 1). We also consider non-economic metrics like death rate and vaccination rate to understand how nations’ varying capabilities to respond to COVID-19 influence their economic performance (idea 2).  \\

\noindent Overall, these findings will help understand the impact COVID-19 exerts on countries from economic and social perspectives. The results may help policy makers at national and international levels (e.g. World Bank) better understand the vulnerability of economies of different nations. Appropriate recommendations can be made to reduce the wealth gap, thus better preparing for future pandemics.
\newpage

\section*{Dataset Description}
\begin{enumerate}
    \item \textbf{Mortality Data}
        \begin{enumerate}[(a)]
            \item \textbf{Source}: World Happiness Report
            \item \textbf{Format}: CSV
            \item \textbf{Relevant Column Headers}: Country name, COVID-19 deaths per 100,000 population in 2020
            \item \textbf{Dimensions}: 167 rows and 17 columns
            \item \textbf{Additional Notes}: None
        \end{enumerate}
    \item \textbf{Annual GDP}
        \begin{enumerate}[(a)]
            \item \textbf{Source}: World Bank
            \item \textbf{Format}: CSV
            \item \textbf{Relevant Column Headers}: Country Name, 1960 GDP, 1961 GDP, …, 2019 GDP, 2020 GDP
            \item \textbf{Dimensions}: 271 rows and 65 columns
            \item \textbf{Additional Notes}: None
        \end{enumerate}
    \item \textbf{Country Income Quartile}
        \begin{enumerate}[(a)]
            \item \textbf{Source}: World Bank
            \item \textbf{Format}: CSV
            \item \textbf{Relevant Column Headers}: Country Code, Income Group
            \item \textbf{Dimensions}: 266 rows and 4 columns
            \item \textbf{Additional Notes}: Income group is separated into 4 quartiles (high, upper middle, lower middle, low)
        \end{enumerate}
    \item \textbf{GDP By Sector}
        \begin{enumerate}[(a)]
            \item \textbf{Source}: World Bank
            \item \textbf{Format}: CSV
            \item \textbf{Relevant Column Headers}: Country Name, Manufacturing 2016, Service 2016, Industry 2016, Agriculture, forestry and fishing 2016 (same format for 2017, 2018, 2019, 2020)
            \item \textbf{Dimensions}: 272 rows and 22 columns
            \item \textbf{Additional Notes}: None
        \end{enumerate}
    \item \textbf{Unemployment Rate}
        \begin{enumerate}[(a)]
            \item \textbf{Source}: World Bank
            \item \textbf{Format}: CSV
            \item \textbf{Relevant Column Headers}: Country Name, 2016 Unemployment as \% of Workforce, 2017 Unemployment as \% of Workforce, 2018 Unemployment as \% of Workforce, 2019 Unemployment as \% of Workforce, 2020 Unemployment as \% of Workforce
            \item \textbf{Dimensions}: 271 rows and 65 columns
            \item \textbf{Additional Notes}: None
        \end{enumerate}
    \item \textbf{Vaccination}
        \begin{enumerate}[(a)]
            \item \textbf{Source}: Our World In Data
            \item \textbf{Format}: CSV
            \item \textbf{Relevant Column Headers}: Total Vaccinations as a Percentage of the Population
            \item \textbf{Dimensions}: 226 rows and 63 columns
            \item \textbf{Additional Notes}: None
        \end{enumerate}
\end{enumerate}

\section*{Computational Plan}
    \subsection*{Data Processing}
    The data needed for the computations are found within separate CSV files. To make the data more uniform and make the code more readable, the data needs to be bundled together and transformed into attributes of class instances. \\

    %TODO: mainly a question for Kai, are we missing any attributes?
    \noindent First, a Country class needs to be defined  with the following attributes:  country name, 2016 - 2020 GDP, 2016 - 2020 GDP by sector, country income quartile, unemployment rate, death rate and vaccination rate. \\

    \noindent Then, a dictionary can be initialized that maps the name of each country to its respective Country instance (that will be initialized with None attributes). With the dictionary filled with key-value pairs, the necessary fields can be extracted from each CSV files and assigned to the appropriate Country instance’s attributes. \\

    \noindent Through this transformation, the dictionary will map the name of each country to a Country instance containing the properties needed to be accessed for computations. We refer to this dictionary as \texttt{country\_dict} in \texttt{Computation By Metrics}.\\

    \subsection*{Required Libraries}
    \begin{enumerate}
        \item \textbf{Plotly}: Plotly is a library that plots data into interactive graphs. Plotly will be used to visualize the data and trends through line graphs for chronological data, bar charts for qualitative data and scatter plots and trendlines to verify relationship between metrics.
        \item \textbf{Geopandas}: Geopandas is a library that helps work with geospatial data. Geopandas will be use to plot choropleth maps which are coloured maps where the colour of a country is determined by a numeric value.
    \end{enumerate}

    \subsection*{Computation By Metrics}
    \begin{enumerate}
        \item \textbf{GDP \% Change}
            \begin{enumerate}[(a)]
                \item \textbf{Formula}: GDP \% Change in Year n $= \frac{\text{GDP}_{n} \ - \ \text{GDP}_{n-1}}{\text{GDP}_{n-1}}$
                \item \textbf{Computation}: For years 2016 to 2020, iterate over all countries in \texttt{country\_dict} and apply the formula for pairs of consecutive years.
                \item \textbf{Visualization}: Generate a chloropleth map from \% change in GDP by year (e.g. dark green represents high GDP growth; dark red represents high GDP deline) using \texttt{Geopandas}. Plot GDP \% change over time as line graphs to see trends using \texttt{Plotly}.
            \end{enumerate}

        \item \textbf{GDP as a \% of Global GDP}
            \begin{enumerate}[(a)]
                \item \textbf{Formula}: GDP of country as \% of Global GDP $= \frac{\text{National GDP}}{\text{Global GDP}}$
                \item \textbf{Computation}: For years 2016 to 2020, iterate over all countries in \texttt{country\_dict} to accumulate their GDP to find global GDP. Then, iterate through all the countries in \texttt{country\_dict} again to use the country's GDP in the given year and the aggregated global GDP and apply the formula.
                \item \textbf{Visualization}: Similar to \texttt{GDP \% Change}
            \end{enumerate}

        \item \textbf{GDP by Sector as a \% of National GDP}
            \begin{enumerate}[(a)]
                \item \textbf{Formula}: GDP of Sector as \% of National GDP $= \frac{\text{Sector GDP}}{\text{National GDP}}$
                \item \textbf{Computation}: For years 2016 to 2020, iterate over all countries in \texttt{country\_dict} and all sectors and calculate GDP of sector as \% of national GDP by using the formula based on the corresponding year.
                \item \textbf{Visualization}: Similar to \texttt{GDP \% Change}.
            \end{enumerate}
        \item Unemployment Rate \% Change
            \begin{enumerate}[(a)]
                \item \textbf{Formula}: Unemployment \% Change in Year n $= \frac{\text{Unemployment \ Rate}_{n}-\  \text{Unemployment \ Rate}_{n-1}}{\text{Unemployment \ Rate}_{n-1}}$
                \item \textbf{Computation}: Iterate over all countries in \texttt{country\_dict} and apply the formula for pairs of consecutive years from 2016 to 2020.
                \item \textbf{Visualization}: Similar to \texttt{GDP \% Change}.
            \end{enumerate}
        \item Sector GDP as a \% of Global GDP Grouped by National Income Quartile
            \begin{enumerate}[(a)]
                \item \textbf{Formula}: Sector value added as \% of Global GDP $= \frac{\text{Aggregated \ Global \ Sector \ Value \ Added \ for \ Quartile}}{\text{Global \ GDP}}$
                \item \textbf{Computation}: For years 2016 to 2020, iterate over \texttt{country\_dict} and accumulate national GDP of all countries to calculate the global GDP and accumulate the sector value added for the nations in the desired quartile to calculate global sector value added for quartile. Then, apply the formula with the values calculated.
                \item \textbf{Visualization}: Generate a bar graph using \texttt{Plotly} and divide it into the four sectors (manufacturing, service, industry, and agriculture, forestry and fishing). For each sector, plot the percentage of global GDP each sector is responsible for by country income quartiles (high, upper-middle, lower-middle, and low).
            \end{enumerate}
        \item COVID-19 related deaths as a predictor of National GDP
            \begin{enumerate}[(a)]
                \item \textbf{Formula}: National GDP = $\beta_{0} + \beta_{1} \times$ COVID-19 deaths per 100,000 population in 2020 $+ \  \epsilon_{i}$
                \item \textbf{Computation}: Using linear regression, compute the fitted linear regression line and the regression coefficients to determine an equation that  predicts National GDP given the COVID-19 deaths per 100,000 population in 2020 of a country.
                \item \textbf{Visualization}: Create a scatter plot with each point being a single country on an x-axis of COVID-19 deaths per 100,000 population in 2020 and a y-axis of National GDP and plot the fitted linear regression line using \texttt{Plotly}.
            \end{enumerate}
        \item Total Vaccinations as a Percentage of the Population as a predictor of National GDP
            \begin{enumerate}[(a)]
                \item \textbf{Formula}: National GDP = $\beta_{0} + \beta_{1} \times$ Total Vaccinations as a Percentage of the Population $+ \ \epsilon_{i}$
                \item \textbf{Computation}: Similar to COVID-19 related deaths as a predictor of National GDP
                \item \textbf{Visualization}: Similar to COVID-19 related deaths as a predictor of National GDP
            \end{enumerate}
    \end{enumerate}
\newpage
\begin{center}
\section*{References}
\end{center}
    \item GDP (current US\$). (n.d.). Retrieved from https://data.worldbank.org/indicator/NY.GDP.MKTP.CD?view=chart\\
    \item Global Inequality. (2021, July 08). Retrieved from https://inequality.org/facts/global-inequality/ \\
    \item Global manufacturing production drops sharply due to economic disruptions caused by COVID-19 – UNIDO report. (n.d.). Retrieved from https://www.unido.org/news/global-manufacturing-production-drops-sharply-due-economic-disruptions-caused-covid-19-unido-report \\
    \item Haseltine, W. A. (2021, March 24). What Can We Learn From Australia's Covid-19 Response? Retrieved from $https://www.forbes.com/sites/williamhaseltine/2021/03/24/what-can-we-learn-from-australias-covid-19-response/?sh=192a771b3a01$ \\
    \item Krieger, T., & Meierrieks, D. (2016, March). Does Income Inequality Lead to Terrorism? (Rep. No. 5821). Retrieved https://www.ifo.de/DocDL/cesifo1\_wp5821.pdf \\
    \item Suneson, G. (2020, March 21). Industries hit hardest by coronavirus in the US include retail, transportation, and travel. Retrieved from https://www.usatoday.com/story/money/2020/03/20/us-industries-being-devastated-by-the-coronavirus-travel-hotels-food/111431804/ \\
    \item Taplin, N. (2021, March 09). Why China Worries About Losing Manufacturing. Retrieved from https://www.wsj.com/\\articles/why-china-worries-about-losing-manufacturing-11615275559 \\
    \item World Development Indicators. (n.d.). Retrieved from https://databank.worldbank.org/source/world-development-indicators?l=en# \\



% NOTE: LaTeX does have a built-in way of generating references automatically,
% but it's a bit tricky to use so we STRONGLY recommend writing your references
% manually, using a standard academic format like APA or MLA.
% (E.g., https://owl.purdue.edu/owl/research_and_citation/apa_style/apa_formatting_and_style_guide/general_format.html)

\end{document}
